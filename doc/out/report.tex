\documentclass{article}
\usepackage[utf8]{inputenc}
\usepackage[T1]{fontenc}
\usepackage[french]{babel}
\usepackage[parfill]{parskip}
\usepackage{amsmath}
\usepackage{amssymb}
\usepackage{amsfonts}
\usepackage{subfigure}
\usepackage[font={small}]{caption}
\usepackage{float}
%\usepackage{listingsutf8}
\usepackage{fullpage}
\usepackage[nochapter]{vhistory}
\usepackage[colorlinks]{hyperref}
\usepackage{titlesec}
\usepackage{xcolor}
\usepackage{verbatim}
\usepackage{graphicx}
\usepackage{comment}

\usepackage{natbib}
\usepackage{url}
\usepackage[compatible]{algpseudocode}

\usepackage{adjustbox}



\newcommand*{\MyIncludeGraphicsMaxSize}[2][]{%
    \begin{adjustbox}{max size={\textwidth}{\textheight}}
        \includegraphics[#1]{#2}%
    \end{adjustbox}
}
\usepackage{array,booktabs,ragged2e}
\usepackage{lstdoc}
\newcolumntype{R}[1]{>{\RaggedLeft\arraybackslash}p{#1}}
\newcolumntype{D}[1]{>{\RaggedLeft\arraybackslash}p{#1}}

% -----------------------------------------------------
% -----------------------------------------------------
% -----------------------------------------------------

\hypersetup{
%couleurs des liens cliquable changée pour une meilleur lisibilité
    colorlinks=true,
    linkcolor=blue,
    filecolor=magenta,
    urlcolor=cyan,
    pdfpagemode=FullScreen,
}

\begin{document}

\title{Flot Maximum}
\author{Thomas Suau, Noé Bourgeois }
\date{March 2023}

\maketitle

\tableofcontents
 
 \newpage

    \section{Introduction}
    Le problème de flot maximum est un problème d'optimisation en théorie des graphes,
    qui consiste à trouver le flot maximum qu'il est possible d'envoyer d'un nœud source à un nœud puits
    dans un graphe pondéré orienté où les arcs ont des capacités maximales.

    Dans ce rapport, nous allons présenter deux méthodes
    pour résoudre le problème de flot maximum.
    La première méthode est la résolution à l'aide d'un programme linéaire avec le solver glpk,
    et la seconde méthode est la méthode des chemins augmentants implémentée en python3.
    
    Dans notre problème les instances sont présentées sous la forme : 
    nodes
    ,mkds
    SLLSLS
    

\section{Programme linéaire}
    
\subsection{Formulation}

On souhaite calculer le flot maximal 


On définit la {\it valeur du flot} v comme la somme
Tout d'abord nous nous appuyons sur le cours qui nous donne la formulation du problème du flot maximal 
Le problème de flot maximum peut être formulé comme un programme linéaire.
       
Soit G=(V, A) un graphe orienté pondéré avec un nœud source s et un nœud puits t.
Soit $f_{ij}$ la quantité de flux circulant sur l'arc $(i,j) \in A$, et $c_{ij}$ la capacité maximale de cet arc.

Le problème de flot maximum peut être formulé comme suit : \\

Maximiser :
$$\sum_{(i,j) \in A} f_{ij}$$
sous les contraintes :

$$\sum_{(i,j) \in A} f_{ij} - \sum_{(j,i) \in A} f_{ji} =
\begin{cases}
-q & \text{si } i=s \\
q & \text{si } i=t \\
0 & \text{sinon}
\end{cases}$$

où $q$ est le flot sortant de la source $s$.


$$f_{ij} \leq c_{ij} ~ \forall (i,j) \in A$$

$$f_{ij} \geq 0 ~ \forall (i,j) \in A $$

Les variables $f_{ij}$ représentent le flot circulant sur l'arc $(i,j)$ et $c_{ij}$ est la capacité maximale de l'arc $(i,j)$.
    
Les contraintes expriment la conservation de flux en chaque nœud, la capacité maximale de chaque arc et le fait que le flot doit être positif.


\section{Chemins augmentants}

— une description de la méthode des chemins augmentant implémentée,


\section{Résultats}
    
 On a réalisé nos tests sur le système d’exploitation : Windows 11 \& MacOS Ventura.\\

— une analyse des résultats de la résolution des différentes instances.
   Cette analyse peut comparer les
temps de résolution
   pour des instances de taille croissante

   pour les deux m´ethodes de r´esolution.

Vous pouvez analyser
   les temps de r´esolution moyens ainsi que

   l’´ecart-type pour les diff´erentes tailles
d’instances.
   Tˆachez d’interpr´eter le pourquoi des r´esultats et

   identifiez si une m´ethode semble plus
appropri´ee.

Un nombre exhaustif d’instances est fournies. Le rapport ne doit pas contenir les r´esultats de toutes
les instances. S´electionnez les instances afin d’ˆetre le plus complet dans vos r´esultats.


%\begin{comment}




    \newpage

    \subsection{Conclusions }



    \newpage



    \section{Ressources}
    \underline{(cf. énoncé)}
    Rédaction scientifique:

    \href{http://informatique.umons.ac.be/algo/redacSci.pdf.}{http://informatique.umons.ac.be/algo/redacSci.pdf.
    }

    Ressources bibliographiques:

    \href{https://www.bibtex.com/.}{https://www.bibtex.com/.}

    L'écriture du code a été accélérée à l'aide du plugin "Github Copilot"
    \href{https://copilot.github.com/}{https://copilot.github.com/}


\end{document}
