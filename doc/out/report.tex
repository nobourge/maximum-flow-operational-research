\documentclass{article}[12pt]
\usepackage[utf8]{inputenc}
\usepackage[T1]{fontenc}
\usepackage[french]{babel}
\usepackage[parfill]{parskip}
\usepackage{amsmath}
\usepackage{amssymb}
\usepackage{amsfonts}
\usepackage{subfigure}
\usepackage[font={small}]{caption}
\usepackage{float}
%\usepackage{listingsutf8}
\usepackage{fullpage}
\usepackage[nochapter]{vhistory}
\usepackage[colorlinks]{hyperref}
\usepackage{titlesec}
\usepackage{xcolor}
\usepackage{verbatim}
\usepackage{graphicx}
\usepackage{comment}
\usepackage{enumitem}
\frenchbsetup{StandardLists=true}

\usepackage{natbib}
\usepackage{url}
\usepackage[compatible]{algpseudocode}

\usepackage{adjustbox}



\newcommand*{\MyIncludeGraphicsMaxSize}[2][]{%
    \begin{adjustbox}{max size={\textwidth}{\textheight}}
        \includegraphics[#1]{#2}%
    \end{adjustbox}
}
\usepackage{array,booktabs,ragged2e}
\usepackage{lstdoc}
\newcolumntype{R}[1]{>{\RaggedLeft\arraybackslash}p{#1}}
\newcolumntype{D}[1]{>{\RaggedLeft\arraybackslash}p{#1}}

% -----------------------------------------------------
% -----------------------------------------------------
% -----------------------------------------------------

\hypersetup{
%couleurs des liens cliquable changée pour une meilleur lisibilité
    colorlinks=true,
    linkcolor=blue,
    filecolor=magenta,
    urlcolor=cyan,
    pdfpagemode=FullScreen,
}

\begin{document}

\title{Flot Maximum}
\subtitle{Projet d'algorithme et recherche opérationnelle 2023}
\author{Thomas Suau, Noé Bourgeois }
\date{March 2023}

\maketitle

 \newpage

\tableofcontents
 
 \newpage

\section{Introduction}

\subsection{Présentation du problème}
    Le problème de flot maximum est un problème d'optimisation en théorie des graphes, qui consiste à trouver le flot maximum qu'il est possible d'envoyer d'un nœud source à un nœud puits dans un graphe pondéré orienté où les arcs ont des capacités maximales. On en donne une expression ci-dessous \ref{math_intro}.\\
    
    Dans ce rapport, nous allons présenter deux méthodes pour résoudre le problème de flot maximum.
    La première méthode est la résolution à l'aide d'un programme linéaire avec le solver glpk, et la seconde méthode est la méthode des chemins augmentants implémentée en python3.
    
    Dans notre problème les instances sont présentées sous la forme : \\
	\begin{lslisting} % Remplacer par algpseudocode qui marche
		nodes n (nombre de noeuds)\\
		source u (numéro de la source)\\
		sink v (numéro du puit)\\
		arcs a (nombre d’arcs)\\
		i j c (présence d’un arc $(i,j)$ et capacité $c_{i,j}$ associée)
		
	\end{lslisting}
\subsection{Outils mathématiques \label{math_intro}}

    Soit $G = (A,N,C)$ un graphe, $s$ sa source, $t$ son puit et dont chaque arc $(i,j)$ est pondéré par sa {\it capacité} $c_{i,j} \in C$ .\\ 
    La capacité pourrait être n'importe quel nombre réel positif mais dans ce cours on se concentre sur des capacités entières.\\
       
    On appelle {\it flot} sur $G$ tout vecteur $(f_{i,j})_{(i,j) \in A}$ vérifiant les deux conditions : \\
    \begin{enumerate}[label=\roman*)]
    	\item $0 \leq f_{i,j} \leq c_{i,j}, ~ \forall (i,j) \in A$ ; \\
	\item  (conservation du flot) $$ ~ \sum_{j : (i,j) \in A} f_{i,j} = \sum_{j : (j,i) \in A} f_{j,i}, ~ \forall i \in N \setminus {(s,t)}$$ \\
    \end{enumerate}
    
    On appelle {\it flot maximal} (ou {\it flot max}) un vecteur $(f_{i,j})_{(i,j)\in A}$ tel que pour tout autre flot $(f'_{i,j})_{(i,j)\in A}$, $f'_{i,j} \leq f_{i,j}$ pour tout $(i,j) \in A$.\\
    
    {\bf Proposition 1} : Pour tout graphe $G$ fini il existe un flot max. \\
    
    {\it Démonstration} : On chercher à prouver L'existence d'un plus grand élément de $\mathcal{F} = \lbrace (f_{i,j})_{(i,j) \in A} | f \text{est un flot} \rbrace$. D'abord, on voit que $\mathcal{F}$ n'est pas vide car le vecteur $(0)_{(i,j) \in A}$ est dans $\mathcal{F}$.\\
    On remarque que $\mathcal{F}$ est fini car $\mathcal{F} \subset \times_{(i,j)\in A} \lbrace 0,...,c_{i,j}$ et $c_{i,j} \in C$ est fini pour tout $(i,j) \in A$. Or tout ensemble fini admet un plus grand et un plus petit élément au sens de l'inclusion. \qed

{\bf Proposition 2} : Pour tout flot $f \in \mathcal{F}$ dans $G$, on a l'égalité : $$ \sum_{j : (s,j) \in A} f_{s,j} = -\sum_{i : (i,t) \in A} f_{i,t}. $$    
    
    {\it Démonstration} : Par principe de conservation du flot $\sum_{j : (i,j) \in A} f_{i,j} = \sum_{j : (j,i) \in A} f_{j,i}$ pour $i \ne s$. Mais en particulier $f_{s,i} - \sum_{j : (i,j) \in A} f_{i,j} = 0$. % A terminer car je trouve que f_{s,i} = f_{t,j} et non pas - f_{t,j}
    
     
    On appelle {\it valeur du flot} $f$, notée $v(f)$, l'élément : $$v(f) = \sum_{(s,j) \in A} f_{s,j} = -\sum_{(i,t) \in A} f_{i,t}$$
    Par le principe de conservation du flot cette valeur existe. Quand on cherche le flot max, on notera simplement $v$ la valeur de flot recherchée. 


\section{Programme linéaire}
    
\subsection{Formulation}

On souhaite calculer le flot maximal via la méthode linéaire en utilisant le solveur \texttt{glpsol}.

On cherche donc à maximiser $v$ définit dans \ref{math_intro}.
Avec les notations de l'introduction soit $f_{i,j}$ la quantité de flot circulant sur l'arc $(i,j) \in A$, et $c_{i,j}$ la capacité maximale de cet arc.

Le problème de flot maximum peut être formulé comme suit : \\
Maximiser :
$$\sum_{(s,j) \in A} f_{s,j}$$
sous les contraintes :
$$\sum_{(i,j) \in A} f_{ij} - \sum_{(j,i) \in A} f_{ji} =
\begin{cases}
-v & \text{si } i=s \\
v & \text{si } i=t \\
0 & \text{sinon}
\end{cases}$$

$$f_{ij} \leq c_{ij} ~ \forall (i,j) \in A$$

$$f_{ij} \geq 0 ~ \forall (i,j) \in A $$

Cette formulation caractérise entièrement notre problème car par la proposition 1, un flot max existe et s'il existe il doit alors vérifier la proposition 2 ; qui nous assure alors que c'est cet élément $\sum_{(s,j) \in A} f_{s,j}$ que l'on doit maximiser. \\
Les contraintes proviennent des conditions pour $f$ d'être un flot.\\

\subsection{Algorithme}

On a considéré des{\it arcs jumeaux} comme étant les arcs ayant même noeud de départ et d'arrivée mais avec des capacités différentes\\
Ainsi, on a notée le flot associé à ces arcs : $f\_i\_j\_c_{ij}$ pour chaque capacité. On ne néglige ainsi aucun arc dans notre résolution mais cela augmente le nombre de variables.\\

{\bf Structure du code } : \\
Il y a un dossier \texttt{Linear} qui contient le fichier \texttt{generate\_model.py}. \\
Les instances se trouvent dans \texttt{./src/Instances/Instances} et \texttt{generate\_model.py} produit les modèles dans \texttt{./src/Instances/model}.
On lance alors depuis le dossier \texttt{src} : 
% Afficher du code bash
\texttt{python3 Linear/generate\_model.py Instances/Instances/inst-100-0.1.txt}
Cela crée le fichier \texttt{Instances/Models/model-100-0.1.lp}.\\
 Afin de résoudre ceci on exécute la commande : \\
 \textt{glpsol - -lp Instances/Models/model-100-0.1.lp -o Instances/Solutions/model-100-0.1.sol}


\section{Chemins augmentants}

— une description de la méthode des chemins augmentant implémentée

Plusieurs algorithme on été choisit et testé.

{\bf Ford-Fulkerson} : \\

Cette algorithme consiste à partir du flot nul $(0)_{(i,j) \in A}$ puis prendre un $st$-chemin, i.e. un chemin de $s$ à $t$.
On sature le $st$-chemin avec le minimum de la capacité sur le chemin choisit.\\
Ensuite, on ... % A terminer  

{\bf } : \\

{\bf } : \\

\section{Résultats}
    
 On a réalisé nos tests sur le système d’exploitation : Windows 11 \& MacOS Ventura.\\

— une analyse des résultats de la résolution des différentes instances.



   pour les deux m´ethodes de r´esolution.

Vous pouvez analyser
   les temps de r´esolution moyens ainsi que

   l’´ecart-type pour les diff´erentes tailles
d’instances.
   Tˆachez d’interpr´eter le pourquoi des r´esultats et

   identifiez si une m´ethode semble plus
appropri´ee.

Un nombre exhaustif d’instances est fournies. Le rapport ne doit pas contenir les r´esultats de toutes
les instances. S´electionnez les instances afin d’ˆetre le plus complet dans vos r´esultats.


%\begin{comment}




    \newpage

    \subsection{Conclusions }



    \newpage
    
    
\begin{thebibliography}{99}

\bibitem{bibtex}{\href{https://www.bibtex.com/}{bibtex}} 

\bibitem{copilot}{L'écriture du code a été accélérée à l'aide du plugin \href{https://copilot.github.com/}{Github Copilot}}

\bibitem{redaction-sci}{\href{http://informatique.umons.ac.be/algo/redacSci.pdf}{Rédaction scientifique}}


    


\end{thebibliography}


    

    


\end{document}
